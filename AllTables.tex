\chapter{Chapter 1} \label{chap1} 

\bgroup
\def\arraystretch{1.2}
\begin{tabular}{lc}
\toprule
\textbf{Classification} & \textbf{BMI} \\ \cmidrule(lr){1-2}
Underweight & \textless{}18.50 \\
Normal range & 18.50--24.99 \\
Overweight & 25.00--29.99 \\
Obese class I & 30.00--34.99 \\
Obese class II & 35.00--39.99 \\
Obese class III & 40.00 or more \\
\bottomrule
\end{tabular}
\egroup


\bgroup
\def\arraystretch{1.2}
\begin{tabular}{lc}
\toprule
\textbf{Risk factor} & \textbf{\% of total DALY} \\ \cmidrule(lr){1-2}
Tobacco use & 9.0 \\
High body mass & 5.5 \\
Alcohol use & 5.1 \\
Physical inactivity & 5.0 \\
High blood pressure & 4.9 \\
High blood plasma glucose & 2.7 \\
High cholesterol & 2.4 \\
Occupational exposures and hazards & 1.9 \\
Drug use & 1.8 \\
Dietary risk factors & 10.4 \\
Joint effect & 31.5 \\
\bottomrule
\end{tabular}
\egroup

\onecolumn
\bgroup
\def\arraystretch{1.5}
\begin{tabularx}{\textwidth}{ll*1{>{\arraybackslash}X}}
\toprule
\multicolumn{1}{p{7cm}}{\textbf{Campaign name}} & \textbf{Year} & \textbf{Description} \\ \cmidrule(lr){1-3}
\multicolumn{1}{p{7cm}}{Healthy Food Partnership} & 2015 & Aims to raise awareness of healthier food choices and portion sizes and to encourage product reformulation. Members include food industry and public health representatives \\
\multicolumn{1}{p{7cm}}{eatforhealth.gov.au} & 2013 & Provides information about healthy eating, including the Australian Dietary Guidelines \\
\multicolumn{1}{p{7cm}}{Weighing it up: Obesity in Australia (House of Representatives Standing Committee on Health and Ageing)} & 2009 & Provides recommendations on what governments, industry, individuals and the community can do to reverse the obesity epidemic and reports on obesity's implications for Australia's health system. \\ 
\multicolumn{1}{p{7cm}}{Australia: The Healthiest Country by 2020 (National Preventative Health Taskforce)} & 2009 & The report outlines ten key actions areas to address obesity, including: increasing the availability and demand for healthier food products, reducing exposure of children to marketing of unhealthy foods and decreasing the availability and demand for unhealthy food products (including through pricing measures). \\
\multicolumn{1}{p{7cm}}{Healthy Weight for Adults and Older Australians} & 2006 & The report outlined three goals; prevent weight gain at the population level, achieve better management of early risk, improve management of weight. \\
\bottomrule
\end{tabularx}
\egroup
\twocolumn

\bgroup
\def\arraystretch{1.2}
\begin{tabular}{lcc}
\toprule
\textbf{Source} & \textbf{Year} & \textbf{Total costs} \\ \cmidrule(lr){1-3}
PwC & 2011/12 & \$8.6 \\
Access Economics & 2008 & \$9.7 \\
Medibank & 2008/09 & \$8.9 \\
Colagiuri \emph{et al } & 2005 & \$12.9 \\
\bottomrule
\end{tabular}
\egroup

\bgroup
\def\arraystretch{1.2}
\begin{tabularx}{\columnwidth}{l*1{>{\arraybackslash}X}}
\toprule
\textbf{Factors} & \textbf{Underlying causes} \\ \cmidrule(lr){1-2}
\textbf{Energy-in} & \begin{tabular}{@{}l@{}} Widespread food marketing\\
Proliferation of cheap, energy-dense foods\\
Increasing palatability of processed food\\
Bigger portion sizes\\
More women working \\
\multicolumn{1}{@{}p{7cm}}{Rising incomes -- leading to more eating out and takeaway food} \end{tabular} \\ \cmidrule(lr){1-2}
\textbf{Energy-out } & \begin{tabular}{@{}l@{}} Sedentary leisure activities\\
Less physically-demanding work\\
Wider car ownership\\ 
Increasing urbanisation \end{tabular}  \\ \cmidrule(lr){1-2}
\textbf{Other } & \begin{tabular}{@{}l@{}} \multicolumn{1}{@{}p{7cm}}{Genetics: a factor for some individuals but not everyone} \\
Rare genetic conditions\\
Epigenetics\\
Greater use of pharmaceuticals\\
Falling smoking rates\\
Too little sleep \end{tabular} \\
\bottomrule
\end{tabularx}
\egroup

\onecolumn
\bgroup
\def\arraystretch{1.2}
\begin{tabularx}{\textwidth}{lll*1{>{\arraybackslash}X}}
\toprule
\textbf{Tax option} & \textbf{Example} & \textbf{Advantages} & \textbf{Disadvantages} \\ \cmidrule(lr){1-4}

\begin{tabular}{@{}l@{}} \multicolumn{1}{@{}p{4.5cm}}{\textbf{Individual tax on excess `empty' calories}} \end{tabular}
 & \begin{tabular}{@{}l@{}} \multicolumn{1}{@{}p{5.5cm}}{Tax above a personalised level of consumption of nutrient poor foods} \end{tabular}
 & \begin{tabular}{@{}l@{}} \multicolumn{1}{@{}p{6.5cm}}{Targets only the additional, empty calories that cause obesity} \end{tabular} 
 & \begin{tabular}{@{}l@{}} Impossible to implement \end{tabular}
 \\ \cmidrule(lr){1-4}

\multicolumn{1}{p{4.5cm}}{\textbf{Tax on a nutrient profile}}
 & \multicolumn{1}{p{5.5cm}}{Tax on low star rating foods}
 & \begin{tabular}{@{}l@{}} \multicolumn{1}{@{}p{6.5cm}}{Can target unhealthy or energy-dense foods} \end{tabular}
 & \begin{tabular}{@{}l@{}} Complex \\
\multicolumn{1}{@{}p{6cm}}{A food index for tax purposes has not been developed} \end{tabular}
 \\ \cmidrule(lr){1-4}

\multicolumn{1}{p{4.5cm}}{\textbf{Tax on ingredient}}
 & \begin{tabular}{@{}l@{}} \multicolumn{1}{@{}p{5.5cm}}{Tax on sugar or fat used in processed foods} \end{tabular}
 &  \begin{tabular}{@{}l@{}} \multicolumn{1}{@{}p{6.5cm}}{Targets problem ingredients} \\ \multicolumn{1}{@{}p{6.5cm}}{Encourages healthy product reformulation} \end{tabular}
& \begin{tabular}{@{}l@{}} \multicolumn{1}{@{}p{6.5cm}}{A single ingredient is not the problem} \\ \multicolumn{1}{@{}p{6.5cm}}{Unhealthy product reformulation} \\ \multicolumn{1}{@{}p{6.5cm}}{May affect core or healthy foods} \end{tabular} \\ \cmidrule(lr){1-4}

\begin{tabular}{@{}l@{}} \multicolumn{1}{@{}p{4.5cm}}{\textbf{Tax on an ingredient within a product}} \end{tabular}
 & \begin{tabular}{@{}l@{}} \multicolumn{1}{@{}p{5.5cm}}{Tax on sugar contained within sugar-sweetened beverages} \end{tabular}
 & \begin{tabular}{@{}l@{}} \multicolumn{1}{@{}p{6.5cm}}{Targets problem products} \\  \multicolumn{1}{@{}p{6.5cm}}{Changes preferences and tastes, encourages substitution} \\  \multicolumn{1}{@{}p{6.5cm}}{Encourages product reformulation} \end{tabular} 
 & \begin{tabular}{@{}l@{}} \multicolumn{1}{@{}p{6.5cm}}{More difficult than taxing a product} \end{tabular}
  \\ \cmidrule(lr){1-4}

\begin{tabular}{@{}l@{}} \multicolumn{1}{@{}p{4.5cm}}{\textbf{Tax on market segment or product}} \end{tabular}
& \begin{tabular}{@{}l@{}} \multicolumn{1}{@{}p{5.5cm}}{Tax on `fast food' or soft drinks} \end{tabular}
& \begin{tabular}{@{}l@{}} \multicolumn{1}{@{}p{6.5cm}}{Easy to implement} \\ \multicolumn{1}{@{}p{6.5cm}}{Can target problem products} \\ \multicolumn{1}{@{}p{6.5cm}}{Encourages substitution} \end{tabular}
& \begin{tabular}{@{}l@{}} \multicolumn{1}{@{}p{6.5cm}}{Hard to classify a certain segment} \\ \multicolumn{1}{@{}p{6.5cm}}{May capture healthy foods or ingredients} \end{tabular} \\
\bottomrule
\end{tabularx}
\egroup
\twocolumn

\onecolumn
\bgroup
\def\arraystretch{1.2}
\begin{tabularx}{\textwidth}{lcl*1{>{\arraybackslash}X}}
\toprule
\begin{tabular}{@{}l@{}}\textbf{Country/region} \end{tabular} & \begin{tabular}{@{}c@{}} \textbf{Year introduced} \end{tabular} & \begin{tabular}{@{}l@{}} \textbf{Tax coverage} \end{tabular} & \begin{tabular}{@{}l@{}} \textbf{SSB tax details} \end{tabular} \\
\midrule

\textbf{Taxes in place}
 & & & \\ \cmidrule(lr){1-4}


Belgium
 & 2016 & \multicolumn{1}{p{4cm}}{Soft drinks (including artificially sweetened)} & \euro{}0.03/litre (A\$0.04/litre) \\

Fiji
 & 2016 & SSBs & A\$0.03/litre \\

Barbados
 & 2015 & SSBs & 10 per cent ad valorem tax \\

Chile
 & 2015 & SSBs & 18 per cent ad valorem tax on SSBs with sugar content above 6.25g/100 mL (10 per cent tax on SSBs with lower sugar content) \\

Mexico
 & 2014 & SSBs & US\$0.01/fl. oz. (US\$0.34/litre) (A\$0.44/litre) \\

Berkeley, California
 & 2014 & SSBs & 1 peso/litre (A\$0.07/litre) \\

Mauritius
 & 2013 & SSBs & MUR 3/100 grams of sugar (\$A0.11/100 grams) contained within SSBs \\

France
 & 2012 & \multicolumn{1}{p{4cm}}{SSBs and artificially sweetened beverages} & \EUR{0.075}/litre (A\$0.11/litre) \\

Hungary
 & 2011 & \multicolumn{1}{p{4cm}}{Soft drinks and energy drinks} & Soft drinks: HUF 7/litre (\$A0.03/litre) (sugar content greater than 8 grams/100mL); selected energy drinks: HUF 250/litre (\$A1.16/litre) \\

Finland
 & 2011 & Soft drinks & \EUR{0.22}/litre (\$A0.31/litre) on soft drinks with more than 0.5 per cent sugar \\

Nauru
 & 2007 & \multicolumn{1}{p{4cm}}{SSBs and flavoured milk} & 30 per cent ad valorem tax \\

French Polynesia
 & 2002 & Soft drinks & CFP 40/litre (A\$0.48/litre) domestic; CFP 60/litre (A\$0.71/litre) imported \\

Samoa
 & 1984 & Soft drinks & WST 0.4/litre (A\$0.21/litre) \\ \midrule


\textbf{Proposed taxes}
 & & ~ & ~ \\ \cmidrule(lr){1-4}

United Kingdom
 & 2018 & SSBs & \pounds{0.18}/litre (A\$0.30/litre) on SSBs with total sugar content above 5g/100 mL; \pounds{0.24}/litre (A\$0.40) SSBs total sugar content above 8g/100 mL \\

Ireland
 & 2018 & SSBs & TBC (in line with UK) \\ 

Portugal
 & 2017 & Soft drinks & \EUR{0.0822}/litre (A\$0.12/litre) on SSBs with sugar content less than 8g/100 mL; \EUR{0.1646}/litre (A\$0.23/litre) SSBs sugar content above 8g/100 mL \\

South Africa
 & 2017 & SSBs & TBC (Treasury recommends a sugar content tax of ZAR 2.29/100 grams of sugar (A\$0.21/100 grams) contained within SSBs) \\

\multicolumn{1}{p{3.5cm}}{Philadelphia City Council, Pennsylvania}
 & 2017 & \multicolumn{1}{p{4cm}}{SSBs and artificially sweetened beverages} & US\$0.015/fl. oz. (US\$0.51/litre) (A\$0.66/litre) \\
\bottomrule
\end{tabularx}
\egroup
\twocolumn

\bgroup
\def\arraystretch{1.5}
\begin{tabularx}{\columnwidth}{l*1{>{\centering\arraybackslash}X}}
\toprule
\textbf{Policy} & \textbf{Overall support (\%)} \\ \cmidrule(lr){1-2}
\multicolumn{1}{p{8cm}}{Traffic light labelling on all packaged foods} & 87 \\
\multicolumn{1}{p{8cm}}{Product reformulation - reduce fat, sugar and salt in processed foods} & 87 \\
\multicolumn{1}{p{8cm}}{Taxing unhealthy foods and using the money for health programs} & 62 \\
\multicolumn{1}{p{8cm}}{Increasing the price of unhealthy foods to reduce the cost of healthy foods} & 71 \\
\multicolumn{1}{p{8cm}}{Taxing soft drinks to reduce the cost of healthy food} & 69 \\
\multicolumn{1}{p{8cm}}{A ban on advertising at times children watch TV} & 83 \\
\multicolumn{1}{p{8cm}}{A total ban on the advertising of unhealthy foods} & 56 \\
\multicolumn{1}{p{8cm}}{Restrict marketing on websites aimed at kids} & 89 \\
\multicolumn{1}{p{8cm}}{Restrict the use of toys and giveaways} & 86 \\
\multicolumn{1}{p{8cm}}{Restrict sponsorship of children's sporting activities} & 71 \\
\bottomrule
\end{tabularx}
\egroup

\onecolumn
\bgroup
\def\arraystretch{1.5}
\begin{tabularx}{\textwidth}{llll*1{>{\arraybackslash}X}}
\toprule
\textbf{Tax option} & \textbf{Example} & \textbf{Advantages} & \textbf{Disadvantages} & \textbf{Existing tax} \\ \cmidrule(lr){1-5}
\multicolumn{1}{p{3cm}}{Specific excise on sugar content of SSB} & \multicolumn{1}{p{4cm}}{40c/100 grams of sugar in SSBs} & \multicolumn{1}{p{6cm}}{Each gram of sugar taxed consistently

Encourages product reformulation

Consumers can shift to less sugary SSBs

Deters bulk buying} & \multicolumn{1}{p{6cm}}{Potentially more complex than a volumetric excise tax

Eroded by inflation} & \multicolumn{1}{p{3.5cm}}{Beer excise tax (\$47.95 per litre of alcohol)} \\
\multicolumn{1}{p{3cm}}{Specific excise on SSB volume -- escalating rates} & \multicolumn{1}{p{4cm}}{20c/litre on SSBs sugar content \textless{}8g/100mL; 40c/litre on SSBs sugar content \textgreater{}8g/100mL} & \multicolumn{1}{p{6cm}}{Encourages product reformulation to reduce sugar content below threshold

Deters bulk buying} & \multicolumn{1}{p{6cm}}{More complex than one standard volumetric rate

Eroded by inflation} & \multicolumn{1}{p{3.5cm}}{Proposed UK soft drink tax} \\ 
\multicolumn{1}{p{3cm}}{Specific excise on SSB volume} & \multicolumn{1}{p{4cm}}{30c/litre tax on SSBs} & \multicolumn{1}{p{6cm}}{Simple to administer

Deters bulk buying} & \multicolumn{1}{p{6cm}}{Eroded by inflation

More tax paid per gram of sugar on low-sugar drinks} & \multicolumn{1}{p{3.5cm}}{Petroleum excise tax (\$0.396 per litre)} \\   

\multicolumn{1}{p{3cm}}{Ad valorem excise tax} & \multicolumn{1}{p{4cm}}{20 per cent tax on the retail value of SSBs} & \multicolumn{1}{p{6cm}}{Keeps pace with inflation

Simple to administer} & \multicolumn{1}{p{6cm}}{Encourages bulk buying and substitution to cheaper drinks

Unpredictable revenues

Undermined by price cuts} & \multicolumn{1}{p{3.5cm}}{Wine equalisation tax

(29\% of the wholesale value of wine)} \\
\bottomrule
\end{tabularx}
\egroup
\twocolumn



\bgroup
\def\arraystretch{1.5}
\begin{tabularx}{\columnwidth}{p{2.5cm}p{2.4cm}p{3.3cm}*1{>{\centering\arraybackslash}X}}
\toprule
\textbf{Source (year)} & \textbf{Tax details} & \textbf{SSB definition} & \textbf{Revenue (\$millions)} \\ \cmidrule(lr){1-4}
Grattan Institute & 40 cents/100 grams of sugar in SSBs & Water-based, non-alcoholic beverages with added sugar & \$520 \\
Grattan Institute & 30 cents/100 grams of sugar in SSBs & As above & \$400 \\
Grattan Institute & 20c/litre SSBs with sugar content \textless{}8g/100mL; 40c/litre sugar content \textgreater{}8g/100mL & As above & \$480 \\
Grattan Institute & 30 cents per litre volumetric excise & As above & \$430 \\
Grattan Institute & 20 per cent ad valorem excise tax & As above & \$550 \\
Parliamentary Budget Office (2016) & 20 per cent ad valorem excise tax & Water-based, non-alcoholic beverages containing natural sugars and/or added caloric sweeteners (\textgreater{}5g of sugar/100mL) & \$550 \\
Veerman \emph{et al.} (2016) & 20 per cent ad valorem excise tax & Non-alcoholic drink with added sugar & \$400+ \\
\bottomrule
\end{tabularx}
\egroup

\onecolumn
\bgroup
\def\arraystretch{1.5}
\begin{tabularx}{\textwidth}{l *5{>{\centering\arraybackslash}X}}
\toprule
\textbf{Beverage}  & \textbf{Sugar content (grams/100mL)} & \textbf{Total sugar}

\textbf{(grams)} & \textbf{Price}

\textbf{(excl. tax)} & \textbf{Tax (\$)} & \textbf{\% increase in retail price} \\ \cmidrule(lr){1-6}
24*375mL pack of Fanta & 11.2 & 1008 & 30.25 & 4.03 & 13.3 \\
2 litre Pepsi & 11.0 & 220 & 3.41 & 0.88 & 25.8 \\
1.25 litre Coles Raspberry & 11.0 & 137.5 & 0.75 & 0.55 & 73.3 \\
250 mL Red Bull & 11.0 & 27.5 & 3.14 & 0.11 & 3.5 \\
375mL Coke & 10.6 & 39.8 & 2.00 & 0.16 & 7.9 \\
2.4 litre Berri apple juice & 10.1 & 242.4 & 3.00 & 0.97 & 32.3 \\
1.5 litre Lipton mango ice tea & 7.0 & 105 & 4.18 & 0.42 & 10.0 \\
600mL Powerade Mountain Blast & 5.8 & 34.8 & 3.00 & 0.14 & 4.6 \\
\bottomrule
\end{tabularx}
\egroup
\twocolumn

\onecolumn
\bgroup
\def\arraystretch{1.5}
\begin{longtable}{p{3cm}p{4cm}p{9.5cm}p{6.5cm}}
\toprule
\textbf{Authors} & \textbf{Study details} & \textbf{Elasticities / effect on consumption} & \textbf{Effect on population weight} \\
\midrule

\multicolumn{4}{l}{\textbf{Australia (modelling and meta-analyses)}} \\ \cmidrule(lr){1-4}
Veerman \emph{et al.} (2016) & Modelled a 20 per cent ad valorem tax on retail price of SSBs (fruit juices, energy drinks, milk-based drinks and cordials excluded) & Estimated 12 per cent decrease in consumption of SSBs due to the tax (using elasticity of soft drinks of -0.63)

Average change in consumption of SSBs from 141 g/day to 124 g/day across the adult male population and from 76 to 67 g/day for women. & The tax estimated to result in a decline in the prevalence of obesity of about 2.7 per cent (0.7 ppt) among men, and 1.2 per cent (0.3 ppt) among women, compared to business as usual. \\
Etil\'e and Sharma (2015) & Modelled a 20 per cent volumetric excise tax and a 20 per cent ad valorem excise tax on SSBs & Elasticity of SSBs estimated to be -0.95. Heavy consumers less elastic demand, but higher health gains (price elasticity $-$2.3 at the median to $-$0.2 at the 95th quantile).

Volumetric excise: reduction in consumption of around 0:6 l/cap/month at the simulated median. Volumetric excise tax more effective at reducing consumption of heavy SSB consumers than ad valorem tax.

Ad valorem excise: reduction in consumption of around 0.6 l/cap/month at the simulated median. & Reductions in body weight greater for heavy consumers of SSBs under volumetric excise tax compared to ad valorem tax \\
Sharma \emph{et al.} (2014) & Modelled a 20 per cent volumetric excise tax and 20 per cent ad valorem excise tax on SSBs & Mean elasticity of SSBs \textasciitilde{} -0.9. Elasticity of soft drinks = -0.63 (controlling for price endogeneity) (-0.89 with exogenous prices). Diet soft drinks -1.01; fruit juice: -1.20; cordial -0.98 (controlling for price endogeneity)

The reduction in consumption is higher under a volumetric tax.

Substitution (a 3.2 per cent increase) towards diet drinks because of the ad valorem tax. & A 20 per cent valoric tax and a 20 cents/litre volumetric tax lead to reductions in bodyweight, of around 0.29kg and 0.41kg per capita for an average consumer of SSBs, respectively. \\
Sacks \emph{et al.} (2011) & Modelled a 10 per cent ad valorem tax on soft drinks, flavoured mineral waters and electrolyte drinks (in addition to tax on other junk foods) & Reduction in energy intake estimated to be14.9 kJ/day for males and 6.7 kJ/day for females & Reduction in body weight due to reduction in SSB consumption \\
Backholer \emph{et al.} (2016) & Meta-analysis of studies (Australia and international) that identify the effects of SSB taxes by socio-economic position & An SSB tax will be modestly financially regressive, with a similar tax burden across socio-economic position (in dollar terms). & An SSB tax will deliver similar benefits across socio-economic position or possibly greater benefits to people within low socio-economic positions \\ \cmidrule(lr){1-4}
\multicolumn{4}{l}{\textbf{Overseas (evaluation)}} \\  \cmidrule(lr){1-4}
Colchero \emph{et al.} (2016) & Evaluation of Mexican SSB tax using recorded food purchase (household) data & Relative to the counterfactual in 2014, purchases of taxed beverages decreased by an average of 6 per cent and decreased at an increasing rate up to a 12 per cent decline by December 2014. Purchases of untaxed beverages were 4 per cent higher than the counterfactual, (mainly bottled plain water). Reductions were higher among low SES households, averaging a 9 per cent decline during 2014, and 17 per cent decrease by December 2014 (compared with pre-tax trends). & \\
Falbe \emph{et al.} (2016) & Questionnaire to evaluate the change in consumption of SSBs after the introduction of the Berkeley SSB tax (four months after implementation) & Consumption of SSBs decreased 21 per cent in Berkeley and increased by 4 per cent in comparison cities. Water consumption increased more in Berkeley (+63 per cent) than in comparison cities (+19 per cent). & \\ \cmidrule(lr){1-4}
\multicolumn{4}{l}{\textbf{Overseas (modelling and meta-analyses)}} \\ \cmidrule(lr){1-4}
Long \emph{et al.} (2015) & Modelled a US\$0.01 per fl. oz SSB excise tax in the USA & Estimated own-price elasticity of demand for SSBs = $-$1.22. Tax estimated to reduce baseline SSB consumption by 20\% & BMI predicted to fall by 0.16 units among youth and 0.08 units among adults in the second year of the tax \\
Ni Mhurchu \emph{et al.} (2014) & Modelling of 20 per cent ad valorem tax on carbonated drinks in New Zealand & Maori and Pacific consumers have more elastic demand. The tax estimated to reduce daily energy intakes by 0.2 per cent (20kJ/day). & ~ \\ 
Zhen \emph{et al.} (2014) & Modelled a half-cent per fluid ounce excise tax on SSBs in the USA, controlling for price endogeneity & Elasticity of regular carbonated soft drinks: -1.035; diet carbonated soft drinks: -0.959; juice drinks: $-$1.192; sports/energy drinks: $-$2.363.

Some increase in consumption of other foods in response to SSB tax, although overall energy intake falls.

Regular and diet soft drinks found to be substitutes & Tax expected to result in weight reductions of 0.37 and 0.16kg/person in 1 year and 0.70 and 0.31kg/person in 10 years for low- and high-income adults respectively. \\
Manyema \emph{et al.} (2014) & Modelled a 20 per cent ad valorem tax on SSBs in South Africa & An own-price elasticity of demand for SSBs of -1.3 was used. A 20 per cent tax estimated to reduce energy intake by about 30kJ per person per day. & Obesity is projected to reduce by 3.8 per cent in men and 2.4 per cent in women. \\
Finkelstein \emph{et al.} (2013) & Modelled a 20 per cent ad valorem tax on SSBs in the USA & Elasticity for energy within SSBs = -0.90 (instrumental variable estimate), -1.32 (exogenous estimates). More inelastic demand among heavy SSB consumers. & SSB tax would result in a decrease in store-bought energy of 24.3~kcal per day per person. This translates into an average weight loss of 1.6 pounds during the first year and a cumulated weight loss of 2.9 pounds in the long run. \\
Miao \emph{et al.} (2013) & Estimates elasticities of foods that may contribute to obesity & Estimated average own-price elasticity of demand for carbonated soft drinks = $-$0.95. Fruit juices = $-$0.87. & \\
Briggs, Mytton, Kehlbacher\emph{, et al.} (2013) & Modelled a 20 per cent ad valorem tax on SSBs in the UK & 20 per cent tax estimated to reduce SSB consumption by 16 per cent. Substitution to diet drinks, tea and coffee, milk, and fruit juice. Greatest substitution to fruit juice and diet soft drinks among lowest income.

The estimated own price elasticity is $-$0.92 for soft drinks and $-$0.81 for SSBs.

SSBs/soft drinks and diet soft drinks found to be substitutes.

Elasticity across income classes similar, although lower income class has more elastic demand for soft drinks. & Estimated to reduce the number of obese adults by 1.3 per cent and the number who are overweight by 0.9 per cent \\
Escobar \emph{et al.} (2013) & Meta-analysis of nine articles & Across the nine studies, pooled own price-elasticity of demand for SSBs = -1.3 & US articles showed that a higher price could lead to a decrease in BMI, and decrease the prevalence of overweight and obesity \\
Powell \emph{et al.} (2013) & Review of US studies on food price elasticity and SSB taxes & Estimated own-price elasticity of demand for SSBs = $-$1.21, soft drinks = $-$0.86. & SSB taxes have a small effect on weight (but studies analysed US state taxes that are relatively small) \\
Briggs, Mytton, Madden\emph{, et al.} (2013) & Modelled a 10 per cent ad valorem tax on SSBs in Ireland & SSB tax estimated to result in a mean reduction in energy intake of 2.1 kcal per person per day & Estimated to reduce prevalence of obesity by 1.3 per cent and prevalence of overweight by a further 0.7 per cent \\
Andreyeva \emph{et al.} (2011) & Modelled a US\$0.01 per fl. oz SSB excise tax in the USA & Used -price elasticity of demand for SSBs = $-$1.2. Estimated reduction in consumption of SSBs of 24 per cent.

SSB tax could potentially could reduce daily per capita caloric intake from SSBs from the current 190--200 calories to 145--150 calories (assuming no substitution to other caloric beverages or food). & Reduction in SSB consumption could translate into significant losses in average body weight -- up to 5 lb/year. \\
Finkelstein \emph{et al.} (2010) & Modelled a 20/40 per cent ad valorem tax on carbonated SSBs only and SSBs in the USA & Carbonated SSBs own-price elasticity = -0.73.

All SSBs elasticity = $-$0.87; $-$0.49 for households in the 50 per cent to 75 per cent income quartile to 0.06 for the 76 per cent to 100 per cent income quartile. & \\
Andreyeva \emph{et al.} (2010) & Review of 160 studies on the price elasticity of demand for food & Own-price elasticity of demand SSBs = $-$0.8; soft drinks = $-$1 & \\
\bottomrule
\end{longtable}
\egroup
\twocolumn

\onecolumn
\bgroup
\def\arraystretch{1.5}
\begin{tabularx}{\textwidth}{p{3cm}p{5cm}p{5cm}p{10cm}}
\toprule
\textbf{Authors} & \textbf{Study details} & \textbf{Elasticities / effect on consumption} & \textbf{Pass through} \\
\midrule
Berardi \emph{et al.} (2016) & Evaluation of French soft drink tax on consumer prices & & Tax fully shifted to soda, almost fully shifted to fruit drinks, incomplete to flavoured waters (6 months after introduction) \\
Grogger (2015) & Evaluation of Mexico's SSB tax using Mexico's Consumer Price Index & & The SSB excise tax of \textasciitilde{}8-10 percent raised the price of regular soda by 12 per cent \\
Bonnet and R\'equillart (2013), & Modelled the effects of an ad valorem and excise tax on French retail soft drink prices & & Retailers passed on to consumers between 60 and 90 per cent of the ad-valorem tax increase, and between 110 and 130 per cent of the excise tax \\
Bergman and Hansen (2010) & Evaluation of Denmark excise tax & & The increase in soft drink excise tax was on average over-shifted, although many retailers did not increase their price at all \\
Bahl \emph{et al.} (2003) & Evaluation of Ireland's soft drink tax levied in the 1970s to the early 1990s & The estimated price elasticity of demand for soft drinks = --1.10 & Under-shifting of tax to retail prices \\
\bottomrule
\end{tabularx}
\egroup
\twocolumn

\bgroup
\def\arraystretch{1.2}
\begin{tabular}{lc}
\toprule
\textbf{Category (BMI)} & \textbf{Number (000s)} \\ \cmidrule(lr){1-2}
Class I (30-34.99) & 3251 \\
Class II (35-39.99) & 1120 \\
Class III (40+) & 572 \\
\textbf{Total} & \textbf{4944} \\
\bottomrule
\end{tabular}
\egroup

\bgroup
\def\arraystretch{1.2}
\begin{tabular}{lc}
\toprule
\textbf{Category} & \textbf{Third-party costs (\$millions)} \\ \cmidrule(lr){1-2}
GPs, specialists, allied health & \$595 \\
Hospital care & \$628 \\
Pharmaceuticals & \$1379 \\
Foregone tax & \$2647 \\
Additional welfare & \$410 \\
\textbf{Total} & \textbf{\$5658} \\
\bottomrule
\end{tabular}
\egroup

\onecolumn
\bgroup
\def\arraystretch{1.5}
\begin{tabularx}{\columnwidth}{l *3{>{\centering\arraybackslash}X}}
\toprule
\textbf{SSB tax option} & \textbf{Average price per litre after tax} & \textbf{Change in consumption after tax (litres/capita/year)} & \textbf{Tax revenue (\$millions)} \\ \cmidrule(lr){1-4}
Sugar content tax (30c/100 grams) & \$2.27 & $-$7 & \$400 \\
Sugar content tax (40c/100 grams) & \$2.36 & $-$9 & \$520 \\
\multicolumn{1}{p{7cm}}{20c/litre on SSBs with sugar content \textless{}8g/100mL; 40c/litre sugar content \textgreater{}8g/100mL} & \$2.32 & $-$9 & \$480 \\
Volumetric tax (30c/litre) & \$2.30 & $-$8 & \$430 \\
Volumetric tax (40c/litre) & \$2.40 & $-$10 & \$550 \\
Ad valorem (20 per cent of retail value) & \$2.40 & $-$10 & \$550 \\
\bottomrule
\end{tabularx}
\egroup
\twocolumn