
\begin{longtable}{Xp{4.3cm}p{10.1cm}p{6.2cm}}
\caption{Summary of overseas studies on SSB taxes -- evaluation studies}\label{tbl:summary-of-overseas-studies-on-SSB-taxes-evaluation} \\
\toprule
\textbf{Authors} & \textbf{Study details} & \textbf{Elasticities / effect on consumption} & \textbf{Effect on population weight} \\
\midrule
\endfirsthead
\caption*{\Cref{tbl:summary-of-overseas-studies-on-SSB-taxes-evaluation}: \nameref{tbl:summary-of-overseas-studies-on-SSB-taxes-evaluation} (continued)} \\
\toprule
\textbf{Authors} & \textbf{Study details} & \textbf{Elasticities / effect on consumption} & \textbf{Effect on population weight} \\
\midrule
\endhead
\bottomrule
\endfoot
\textcite{Colchero2016Beveragepurchasesstores} & Evaluation of Mexican SSB tax using recorded food purchase (household) data & Relative to the counterfactual in 2014, purchases of taxed beverages decreased by an average of 6 per cent and decreased at an increasing rate up to a 12 per cent decline by December 2014. Purchases of untaxed beverages were 4 per cent higher than the counterfactual, (mainly bottled plain water). Reductions were higher among low SES households, averaging a 9 per cent decline during 2014, and 17 per cent decrease by December 2014 (compared with pre-tax trends). & \\
\textcite{Falbe2016ImpactBerkeleyExcise} & Questionnaire to evaluate the change in consumption of SSBs after the introduction of the Berkeley SSB tax (four months after implementation) & Consumption of SSBs decreased 21 per cent in Berkeley and increased by 4 per cent in comparison cities. Water consumption increased more in Berkeley (+63 per cent) than in comparison cities (+19 per cent). & \\ 
\textcite{Fletcher2010Cansoftdrink} & Evaluation of existing soft drink taxes in US states on weight & & A one percentage point increase in (non-health focused) soft drink taxes in US states decreases adult BMI by 0.003. 
\end{longtable}