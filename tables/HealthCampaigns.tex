\bgroup
%\def\arraystretch{1.8} % makes headers uneven vertically
\def\rowSpace{-2pt}
% Use phantom rows (better solutions welcome)
\begin{tabularx}{\columnwidth}{p{2.9cm}l>{\arraybackslash}X}
\toprule
{\textbf{Campaign name}} & \textbf{Year} & \textbf{Description} \\ 
\midrule
% mbox to override hyphenation
{Healthy Food \mbox{Partnership}} & 2015 & Aims to raise awareness of healthier food choices and portion sizes and to encourage product reformulation. Members include food industry and public health representatives \\ \relax \null & & \\[\rowSpace]
{eatforhealth.gov.au} & 2013 & Provides information about healthy eating, including the Australian Dietary Guidelines \\ \relax \null & & \\[\rowSpace]
{Weighing it up: Obesity in Australia (House of Representatives Standing Committee on Health and Ageing)} & 2009 & Provides recommendations on what governments, industry, individuals and the community can do to reverse the obesity epidemic and reports on obesity's implications for Australia's health system. \\ \relax \null & & \\[\rowSpace] 
{Australia: The Healthiest Country by 2020 (National Preventative Health Taskforce)} & 2009 & The report outlines ten key areas to address obesity, including: increasing the availability and demand for healthier foods, reducing exposure of children to marketing of unhealthy foods and decreasing the availability and demand for unhealthy foods (including through pricing measures). \\ \relax \null & & \\[\rowSpace]
{Healthy Weight for Adults and Older Australians} & 2006 & The report outlined three goals; prevent weight gain at the population level, achieve better management of early risk, improve management of weight. \\ 
\bottomrule
\end{tabularx}
\egroup