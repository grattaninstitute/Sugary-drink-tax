\begin{longtable}{Xp{4.5cm}p{9.9cm}p{6.2cm}}
\caption{Summary of overseas studies on SSB taxes -- modelling and meta-analyses studies}\label{tbl:summary-of-overseas-studies-on-SSB-taxes-modelling-meta-analysis} \\
\toprule
\textbf{Authors} & \textbf{Study details} & \textbf{Elasticities / effect on consumption} & \textbf{Effect on population weight} \\
\midrule
\endfirsthead
\caption*{\Cref{tbl:summary-of-overseas-studies-on-SSB-taxes-modelling-meta-analysis}: \nameref{tbl:summary-of-overseas-studies-on-SSB-taxes-modelling-meta-analysis} (continued)} \\
\toprule
\textbf{Authors} & \textbf{Study details} & \textbf{Elasticities / effect on consumption} & \textbf{Effect on population weight} \\
\midrule
\endhead
\bottomrule
\endfoot
\textcite{Long2015Costeffectivenesssugar} & Modelled a US\$0.01 per fl.\,oz SSB excise tax in the USA & Estimated own-price elasticity of demand for SSBs = $-$1.22. Tax estimated to reduce baseline SSB consumption by 20 per cent & BMI predicted to fall by 0.16 units among youth and 0.08 units among adults in the second year of the tax \\
\textcite{NiMhurchu2014Twentypercenttax} & Modelling of 20\% ad valorem tax on carbonated drinks in New Zealand & Maori and Pacific consumers have more elastic demand. The tax estimated to reduce daily energy intake by 0.2 per cent (20kJ/day). & ~ \\ 
\textcite{Zhen2014Predictingeffectssugar} & Modelled a half-cent per fluid ounce excise tax on SSBs in the USA, controlling for price endogeneity & Elasticity of regular carbonated soft drinks: \(-1.035\); diet carbonated soft drinks: \(-0.959\); juice drinks: $-$1.192; sports/energy drinks: $-$2.363.

Some increase in consumption of other foods in response to SSB tax, although overall energy intake falls.

Regular and diet soft drinks found to be substitutes & Tax expected to result in weight reductions of 0.37 and 0.16kg/person in 1 year and 0.70 and 0.31kg/person in 10 years for low- and high-income adults respectively. \\
\textcite{Manyema2014potentialimpact20} & Modelled a 20\% ad valorem tax on SSBs in South Africa & An own-price elasticity of demand for SSBs of \(-1.3\) was used. A 20\% tax estimated to reduce energy intake by about 30kJ per person per day. & Obesity is projected to reduce by 3.8\% in men and 2.4\% in women. \\
\textcite{Briggs2013Overallincomespecific} & Modelled a 10\% ad valorem tax on SSBs in Ireland & SSB tax estimated to result in a mean reduction in energy intake of 2.1 kcal per person per day & Estimated to reduce prevalence of obesity by 1.3\% and prevalence of overweight by a further 0.7\% \\
\textcite{Briggs2013potentialimpactobesity} & Modelled a 20\% ad valorem tax on SSBs in the UK & 20\% tax estimated to reduce SSB consumption by 16\%. Substitution to diet drinks, tea and coffee, milk, and fruit juice. Greatest substitution to fruit juice and diet soft drinks among lowest income.

The estimated own price elasticity is $-$0.92 for soft drinks and $-$0.81 for SSBs.

SSBs/soft drinks and diet soft drinks found to be substitutes.

Elasticity across income classes similar, although lower income class has more elastic demand for soft drinks. & Estimated to reduce the number of obese adults by 1.3\% and the number who are overweight by 0.9\% \\

\textcite{Escobar2013Evidencethattax} & Meta-analysis of nine articles on SSB taxes & Across the nine studies, pooled own price-elasticity of demand for SSBs = \(-1.3\) & US articles showed that a higher price could lead to a decrease in BMI, and decrease the prevalence of overweight and obesity \\
\textcite{Finkelstein2013Implicationssugarsweetened} & Modelled a 20\% ad valorem tax on SSBs in the USA & Elasticity for energy within SSBs = \(-0.90\) (instrumental variable estimate), \(-1.32\) (exogenous estimates). More inelastic demand among heavy SSB consumers. & SSB tax would result in a decrease in store-bought energy of 24.3~kcal per day per person. This translates into an average weight loss of 1.6 pounds during the first year and a cumulated weight loss of 2.9 pounds in the long run. \\
\textcite{Miao2013Accountingproductsubstitution} & Estimates elasticities of foods that may contribute to obesity & Estimated average own-price elasticity of demand for carbonated soft drinks = $-$0.95. Fruit juices = $-$0.87. & \\
\textcite{Powell2013Assessingpotentialeffectiveness} & Review of US studies on food price elasticity and SSB taxes & Estimated own-price elasticity of demand for SSBs = $-$1.21, soft drinks = $-$0.86. & SSB taxes have a small effect on weight (but studies analysed US state taxes that are relatively small) \\
\textcite{Andreyeva2011Estimatingpotentialtaxes} & Modelled a US\$0.01 per fl.\,oz SSB excise tax in the USA & Used price elasticity of demand for SSBs = $-$1.2. Estimated reduction in consumption of SSBs of 24\%.

SSB tax could potentially could reduce daily per capita caloric intake from SSBs from the current 190--200 calories to 145--150 calories (assuming no substitution to other caloric beverages or food). & Reduction in SSB consumption could translate into significant losses in average body weight -- up to 5 lb/year. \\
\textcite{Finkelstein2010EconomicsObesity} & Modelled a 20/40\% ad valorem tax on carbonated SSBs only and SSBs in the USA & Carbonated SSBs own-price elasticity = \(-0.73\).

All SSBs elasticity = $-$0.87; $-$0.49 for households in the 50\% to 75\% income quartile to 0.06 for the 76\% to 100\% income quartile. & \\
\textcite{Andreyeva2010impactfoodprices} & Review of 160 studies on the price elasticity of demand for food & Own-price elasticity of demand SSBs = $-$0.8; soft drinks = $-$1
\end{longtable}