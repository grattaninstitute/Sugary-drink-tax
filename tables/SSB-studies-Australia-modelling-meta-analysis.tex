\begin{longtable}{Xp{4.3cm}p{10.1cm}p{6.2cm}}
\caption{Summary of Australian studies on SSB taxes}\label{tbl:summary-of-Aust-studies-on-SSB-taxes} \\
\toprule
\textbf{Authors} & \textbf{Study details} & \textbf{Elasticities / effect on consumption} & \textbf{Effect on population weight} \\
\midrule
\endfirsthead
\caption*{\Cref{tbl:summary-of-Aust-studies-on-SSB-taxes}: \nameref{tbl:summary-of-Aust-studies-on-SSB-taxes} (continued)} \\
\toprule
\textbf{Authors} & \textbf{Study details} & \textbf{Elasticities / effect on consumption} & \textbf{Effect on population weight} \\
\midrule
\endhead
\bottomrule
\endfoot
%\multicolumn{4}{c}{\textbf{Australia (modelling and meta-analyses)}} \\ \cmidrule(lr){1-4}
\textcite{Backholer2016impacttaxsugar} & Meta-analysis of studies (Australia and international) that identify the effects of SSB taxes by socio-economic position & An SSB tax will be modestly financially regressive, with a similar tax burden across socio-economic position (in dollar terms). & An SSB tax will deliver similar benefits across socio-economic position or possibly greater benefits to people within low socio-economic positions \\

\textcite{Veerman2016ImpactTaxSugar} & Modelled a 20 per cent ad valorem tax on retail price of SSBs (fruit juices, energy drinks, milk-based drinks and cordials excluded) & Estimated 12 per cent decrease in consumption of SSBs due to the tax (using elasticity of soft drinks of \(-0.63\)).

Average change in consumption of SSBs from 141 g/day to 124 g/day across the adult male population and from 76 to 67 g/day for women. & The tax estimated to result in a decline in the prevalence of obesity of about 2.7 per cent (0.7 ppt) among men, and 1.2 per cent (0.3 ppt) among women, compared to business as usual. \\

\textcite{Yang2016child} & Examined the impact of price changes on children’s consumption of SSBs using
stated preference panel data &  Uncompensated own-price
elasticities for SSBs range from \(-0.83\) to \(-0.94\). 

Low income households are more responsive to price changes. High-income households are less responsive to price and not responsive to non-price attributes. &  \\

\textcite{Etile2015DoHighConsumers}  & Modelled a 20 per cent volumetric excise tax and a 20 per cent ad valorem excise tax on SSBs & Elasticity of SSBs estimated to be \(-0.95\). Heavy consumers are estiamted to have less elastic demand, but higher health gains (price elasticity $-$2.3 at the median to $-$0.2 at the 95th quantile).

Volumetric excise: reduction in consumption of around 0:6 l/cap/month at the simulated median. Volumetric excise tax more effective at reducing consumption of heavy SSB consumers than ad valorem tax.

Ad valorem excise: reduction in consumption of around 0.6 l/cap/month at the simulated median. & Reductions in body weight greater for heavy consumers of SSBs under volumetric excise tax compared to ad valorem tax \\
\textcite{Sharma2014effectstaxingsugarsweetened} & Modelled a 20 per cent volumetric excise tax and 20 per cent ad valorem excise tax on SSBs & Mean elasticity of demand for SSBs approximately equal to \(-0.9\). Elasticity of soft drinks = \(-0.63\) (controlling for price endogeneity) (\(-0.89\) with exogenous prices). Elasticity of diet soft drinks: \(-1.01\); fruit juice: \(-1.20\); cordial \(-0.98\) (controlling for price endogeneity).

The reduction in consumption is higher under a volumetric tax.

Substitution (a 3.2 per cent increase) towards diet drinks because of the ad valorem tax. & A 20 per cent valoric tax and a 20 cents/litre volumetric tax lead to reductions in bodyweight, of around 0.29kg and 0.41kg per capita for an average consumer of SSBs, respectively. \\
\textcite{Sacks2011Statesshouldstand} & Modelled a 10 per cent ad valorem tax on soft drinks, flavoured mineral waters and electrolyte drinks (in addition to tax on other junk foods) & Reduction in energy intake estimated to be 14.9 kJ/day for males and 6.7 kJ/day for females & Reduction in body weight due to reduction in SSB consumption \\

\end{longtable}